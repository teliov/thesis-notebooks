%%%%%%%%%%%%%%%%%%%%%%%%%%%%%%%%%%%%%%%%%
% Beamer Presentation
% LaTeX Template
% Version 1.0 (10/11/12)
%
% This template has been downloaded from:
% http://www.LaTeXTemplates.com
%
% License:
% CC BY-NC-SA 3.0 (http://creativecommons.org/licenses/by-nc-sa/3.0/)
%
%%%%%%%%%%%%%%%%%%%%%%%%%%%%%%%%%%%%%%%%%

%----------------------------------------------------------------------------------------
%	PACKAGES AND THEMES
%----------------------------------------------------------------------------------------

\documentclass{beamer}

\mode<presentation> {

% The Beamer class comes with a number of default slide themes
% which change the colors and layouts of slides. Below this is a list
% of all the themes, uncomment each in turn to see what they look like.

%\usetheme{default}
%\usetheme{AnnArbor}
%\usetheme{Antibes}
%\usetheme{Bergen}
%\usetheme{Berkeley}
%\usetheme{Berlin}
%\usetheme{Boadilla}
%\usetheme{CambridgeUS}
%\usetheme{Copenhagen}
%\usetheme{Darmstadt}
%\usetheme{Dresden}
%\usetheme{Frankfurt}
%\usetheme{Goettingen}
%\usetheme{Hannover}
%\usetheme{Ilmenau}
%\usetheme{JuanLesPins}
%\usetheme{Luebeck}
\usetheme{Madrid}
%\usetheme{Malmoe}
%\usetheme{Marburg}
%\usetheme{Montpellier}
%\usetheme{PaloAlto}
%\usetheme{Pittsburgh}
%\usetheme{Rochester}
%\usetheme{Singapore}
%\usetheme{Szeged}
%\usetheme{Warsaw}

% As well as themes, the Beamer class has a number of color themes
% for any slide theme. Uncomment each of these in turn to see how it
% changes the colors of your current slide theme.

%\usecolortheme{albatross}
%\usecolortheme{beaver}
%\usecolortheme{beetle}
%\usecolortheme{crane}
%\usecolortheme{dolphin}
%\usecolortheme{dove}
%\usecolortheme{fly}
%\usecolortheme{lily}
%\usecolortheme{orchid}
%\usecolortheme{rose}
%\usecolortheme{seagull}
%\usecolortheme{seahorse}
%\usecolortheme{whale}
%\usecolortheme{wolverine}

%\setbeamertemplate{footline} % To remove the footer line in all slides uncomment this line
%\setbeamertemplate{footline}[page number] % To replace the footer line in all slides with a simple slide count uncomment this line

%\setbeamertemplate{navigation symbols}{} % To remove the navigation symbols from the bottom of all slides uncomment this line
}

\usepackage{graphicx} % Allows including images
\usepackage{booktabs} % Allows the use of \toprule, \midrule and \bottomrule in tables

%----------------------------------------------------------------------------------------
%	TITLE PAGE
%----------------------------------------------------------------------------------------

\title[An Update]{Differential Diagnosis: Baseline Analysis and Simulating Real Data} % The short title appears at the bottom of every slide, the full title is only on the title page

\author{Obinna Stanley Agba} % Your name
\institute[TU Delft] % Your institution as it will appear on the bottom of every slide, may be shorthand to save space
{
Delft University of Technology \\ % Your institution for the title page
\medskip
\textit{o.s.agba@student.tudelft.nl} % Your email address
}
\date{\today} % Date, can be changed to a custom date

\begin{document}

\begin{frame}
\titlepage % Print the title page as the first slide
\end{frame}

\begin{frame}
\frametitle{Overview} % Table of contents slide, comment this block out to remove it
\tableofcontents % Throughout your presentation, if you choose to use \section{} and \subsection{} commands, these will automatically be printed on this slide as an overview of your presentation
\end{frame}

%----------------------------------------------------------------------------------------
%	PRESENTATION SLIDES
%----------------------------------------------------------------------------------------

%------------------------------------------------
\section{Baseline Analysis} % Sections can be created in order to organize your presentation into discrete blocks, all sections and subsections are automatically printed in the table of contents as an overview of the talk
%------------------------------------------------
\subsection{Baseline Data Generation}
\begin{frame}
\frametitle{Sample Space}
\begin{itemize}
	\item 801 Conditions i.e diseases
	\item 376 Symptoms
	\item Assumption: these conditions and symptoms capture the "illness" space
	\item Age, Gender and Race also captured
\end{itemize}
\end{frame}

\begin{frame}
\frametitle{Baseline Data Generation}
\begin{itemize}
	\item Generate Synthea compatible modules from Symcat data
	\item Use generated modules with Synthea Generator
	\item Use Symcat data as is i.e no modifications, plug and play
	\item Generated 5 Million conditions for the baseline
\end{itemize}
\end{frame}

\subsection{Naive Bayes on Baseline} % A subsection can be created just before a set of slides with a common theme to further break down your presentation into chunks
\begin{frame}
\frametitle{Naive Bayes on Baseline}
\begin{itemize}
	\item Naive Bayes (N.B.) assumes conditional independence among features
	\item Features in this case include gender, age, race and all symptoms
	\item No hyper-parameters to be optimized
\end{itemize}
\end{frame}

\begin{frame}
\frametitle{Learning Curve (accuracy): Naive Bayes on Baseline}
\begin{figure}
	\includegraphics[width=0.7\linewidth]{figs/nb_lc_baseline.pdf}
	\caption{Learning Curve (Accuracy) for Naive Bayes}
\end{figure}
\end{frame}

\begin{frame}
\frametitle{Learning Curve (precision): Naive Bayes on Baseline}
\begin{figure}
	\includegraphics[width=0.7\linewidth]{figs/nb_lc_prec_baseline.pdf}
	\caption{Learning Curve (Precision) for Naive Bayes}
\end{figure}
\end{frame}

\begin{frame}
\frametitle{Naive Bayes on Baseline: Summary }
\begin{table}[]
	\begin{tabular}{|l|l|l|}
		\hline
		Metric               & Train  & Validation \\ \hline
		Accuracy             & 0.5897 & 0.5893     \\ \hline
		Precision Weighted   & 0.6743 & 0.6773     \\ \hline
		Recall Weighted      & 0.5897 & 0.5893     \\ \hline
		Top 5 Accuracy       & 0.8416 & 0.8411     \\ \hline
	\end{tabular}
	\caption{Baseline Naive Bayes Metric Summary}
	\label{table:tab_1}
\end{table}
\end{frame}

\subsection{Random Forest on Baseline} % A subsection can be created just before a set of slides with a common theme to further break down your presentation into chunks
\begin{frame}
\frametitle{Random Forest on Baseline}
\begin{itemize}
	\item Random Forest (RF) uses an ensemble of trees to learn different "\textit{selections}" of the data
	\item Trees vote on the most probable class
	\item Using sci-kit learn implementation - quite a number of hyper parameters to be optimized
\end{itemize}
\end{frame}

\begin{frame}
\frametitle{Random Forest on Baseline: Hyper-parameter Optimization}
\begin{itemize}
	\item Run a Grid Search on Hyper Parameter Space
	\item Aim is to maximize validation score with "\textit{reasonable}" model size (in MB) and train time (seconds)
	\item Targets for consideration must be within 1\% of best validation score
	\item Select optimal parameter combination using a weighted metric and maximize:
	$$
	metric = 2 * test\_score - 0.1*train\_time - 0.1 * model\_size 
	$$
	\item All parameters have been normalized
\end{itemize}
\end{frame}

\begin{frame}
\frametitle{Random Forest on Baseline: Hyper-parameter Optimization}
\begin{figure}
	\includegraphics[width=0.6\linewidth]{figs/rf_hyper_pareto_val_size.pdf}
	\caption{Random Forest Hyper-Parameter Optimization: Validation Score vs Model Size}
\end{figure}
\end{frame}

\begin{frame}
\frametitle{Random Forest on Baseline: Hyper-parameter Optimization}
\begin{figure}
	\includegraphics[width=0.6\linewidth]{figs/rf_hyper_pareto_val_time.pdf}
	\caption{Random Forest Hyper-Parameter Optimization: Validation Score vs Time}
\end{figure}
\end{frame}

\begin{frame}
\frametitle{Random Forest on Baseline: Hyper-parameter Optimization}
\begin{itemize}
	\item Previous slides show that larger trees which in turn take longer to train do not produce best results
	\item Also after a point, improvement in accuracy is no longer proportional to increased model size/ train time
	\item Using weighted metric, hyper-parameter set selected was within 0.5\% of best validation score but with a 32\% reduction in model size and 59\% reduction in train time.
\end{itemize}
\end{frame}

\begin{frame}
\frametitle{Learning Curve (accuracy): Random Forest on Baseline}
\begin{figure}
	\includegraphics[width=0.7\linewidth]{figs/rf_lc_baseline.pdf}
	\caption{Learning Curve (Accuracy) for Random Forest}
\end{figure}
\end{frame}

\begin{frame}
\frametitle{Learning Curve (precision): Random Forest on Baseline}
\begin{figure}
\includegraphics[width=0.7\linewidth]{figs/rf_lc_prec_baseline.pdf}
\caption{Learning Curve (Precision) for Random Forest}
\end{figure}
\end{frame}

\begin{frame}
\frametitle{Random Forest on Baseline: Summary }
\begin{table}[]
	\begin{tabular}{|l|l|l|}
		\hline
		Metric             & Train  & Validation \\ \hline
		Accuracy           & 0.6378 & 0.5927     \\ \hline
		Precision Weighted & 0.6841 & 0.6281     \\ \hline
		Recall Weighted    & 0.6378 & 0.5927     \\ \hline
		Top 5 Accuracy     & 0.8829 & 0.8539     \\ \hline
	\end{tabular}
	\caption{Baseline Random Forest Metric Summary}
	\label{table:tab_2}
\end{table}
\end{frame}

\section{Beyond the Baseline}
%------------------------------------------------
\subsection{Real Data Behavior?}
\begin{frame}
\frametitle{Real Data Behavior}
\begin{itemize}
	\item How might real data behave?
	\item Are there alterations to the generation process that might simulate conditions in reality?
	\item Observe effects of this alterations to the performance of the models
	\item Rationale: In the end, format of data is ideal. What would change in a real data set would be symptom-condition probabilistic relationships
\end{itemize}
\end{frame}

\begin{frame}
\frametitle{Approaches}
\begin{itemize}
	\item Combine the Condition-Symptom probabilities in a different manner
	\item Randomly perturb the Symtom probabilities - models should be robust to relatively small perturbations in probabilities
	\item Increase the minimum number of symptoms each condition must present
	\item Inject \textit{"similar"} symptoms into conditions
\end{itemize}
\end{frame}


\begin{frame}
\frametitle{Alter Formulation of Condition-Symptom Probabilities}
\begin{itemize}
	\item Baseline uses \textit{"plug and play"} approach to factoring in the Symcat provided probabilities for conditions and symptoms
	\item $$
		P_{condition | age, gender, race} = P_{age|condition} * P_{gender|condition} * P_{race|condition}
	$$
	\item $P_{symptom|condition}$ is plugged in \textit{as-is} into the  Synthea generator
	\item But using demographic data it is possible to obtain $P_{age}$, $P_{gender}$, $P_{race}$
	\item Then using an application of the Bayes rule we can obtain a less direct formulation of $P_{condition | age, gender, race}$
\end{itemize}
\end{frame}


\begin{frame}
\frametitle{Alter Formulation of Condition-Symptom Probabilities}
\begin{itemize}
	\item Symcat also provides information about $P_{symptom|age}, P_{symptom|gender}, P_{symptom|race}$
	\item Using similar methodology, we can use $P_{symptom|condition, age, gender, race}$ instead of just $P_{symptom|condition}$
	\item With these new formulations of the probabilities in Synthea, a different dataset can be generated
\end{itemize}
\end{frame}


\begin{frame}
\frametitle{Perturb Condition-Symptom Probabilities}
\begin{itemize}
	\item Randomly increase/decrease Symptom probabilities by given percentages
	\item Generate data using 10\%, 20\% and 30\% perturbation of symptom probabilities
\end{itemize}
\end{frame}


\begin{frame}
\frametitle{Minimum Number of Symptoms}
\begin{itemize}
	\item During data generation, enforce a minimum number of symptoms that must be presented by a condition
	\item Vary minimum number of symptoms from 1 (baseline) to 5 (average number of symptoms per condition /2)
\end{itemize}
\end{frame}


\begin{frame}
\frametitle{Symptom Injection}
\begin{itemize}
	\item Using a graph based approach, inject \textit{"most likely"} symptoms into a condition which are not already present
	\item The similarity of a symptoms (nodes) is measured by how many conditions (edges) they have in common
	\item For a condition, the most similar symptom would be one which has the highest number of connections to original symptoms of the condition
	\item But is not included as a symptom of the condition.
	\item As an example, for \textit{Pharyngitis} the following additional symptoms were injected: 
	\begin{itemize}
		\item Vomiting, Headache, Difficulty in breathing
		\item Shortness of Breath, Sharp Chest Pain
	\end{itemize}
	\item Injected symptoms are assigned probabilities lower than that of the lowest symptom
\end{itemize}
\end{frame}

\subsection{Results on Newly Generated Data}
\begin{frame}
\frametitle{Altered Formulation of Condition-Symptom Prob.: Summary }
\begin{table}[]
	\begin{tabular}{|l|l|l|l|l|}
		\hline
		Metric             & NB Train & NB Val & RF Train  & RF Val \\ \hline
		Accuracy         & 0.5913 & 0.5907   &0.6420 & 0.6035     \\ \hline
		Precision Weighted & 0.6670 & 0.6660 &0.6879 & 0.6391     \\ \hline
		Recall Weighted     & 0.5914 & 0.5908 & 0.6420 & 0.6035    \\ \hline
		Top 5 Accuracy      & 0.8478 & 0.8472 & 0.8848 &  0.8623    \\ \hline
	\end{tabular}
	\caption{Altered Condition-Symptom Prob. Metric Summary}
	\label{table:tab_3}
\end{table}
\end{frame}

\begin{frame}
\frametitle{Perturbed (30\%) Symptom Prob.: Summary }
\begin{table}[]
	\begin{tabular}{|l|l|l|l|l|}
		\hline
		Metric             & NB Train & NB Val & RF Train  & RF Val \\ \hline
		Accuracy         & 0.5967 & 0.5962   &0.6207 & 0.5842     \\ \hline
		Precision Weighted & 0.6553 &  0.6548 &0.6698 & 0.6249     \\ \hline
		Recall Weighted     & 0.5967 & 0.5962 & 0.6207 & 0.5842    \\ \hline
		Top 5 Accuracy      &0.8610 & 0.8616 &0.8793 &  0.8570    \\ \hline
	\end{tabular}
	\caption{Perturbed (30\%) Symptom Prob. Metric Summary}
	\label{table:tab_4}
\end{table}
\end{frame}

\begin{frame}
\frametitle{Minimum of 3 Symptoms: Summary }
\begin{table}[]
	\begin{tabular}{|l|l|l|l|l|}
		\hline
		Metric             & NB Train & NB Val & RF Train  & RF Val \\ \hline
		Accuracy         & 0.7871 & 0.7867  & 0.8189 & 0.7849    \\ \hline
		Precision Weighted & 0.8006 &  0.8002 &0.8301 &0.7914    \\ \hline
		Recall Weighted     & 0.7871 & 0.7867 & 0.8189 & 0.7848    \\ \hline
		Top 5 Accuracy      &0.9634 & 0.9631 &0.9780 &  0.9621    \\ \hline
	\end{tabular}
	\caption{Minimum of 3 Symptoms. Metric Summary}
	\label{table:tab_5}
\end{table}
\end{frame}

\begin{frame}
\frametitle{Minimum of 5 Symptoms: Summary }
\begin{table}[]
	\begin{tabular}{|l|l|l|l|l|}
		\hline
		Metric             & NB Train & NB Val & RF Train  & RF Val \\ \hline
		Accuracy         &0.9137 & 0.9135  & 0.9324 & 0.9129   \\ \hline
		Precision Weighted & 0.9252 &  0.9250 &0.9351 &0.9148   \\ \hline
		Recall Weighted     &  0.9136 &0.9135 & 0.9324 & 0.9129   \\ \hline
		Top 5 Accuracy      &0.9888 & 0.9886 &0.9961 &  0.9878   \\ \hline
	\end{tabular}
	\caption{Minimum of 5 Symptoms. Metric Summary}
	\label{table:tab_6}
\end{table}
\end{frame}

\begin{frame}
\frametitle{Injected Symptoms: Summary }
\begin{table}[]
	\begin{tabular}{|l|l|l|l|l|}
		\hline
		Metric             & NB Train & NB Val & RF Train  & RF Val \\ \hline
		Accuracy         &0.5544 &0.5522 & 0.9324 & 0.9129   \\ \hline
		Precision Weighted & 0.5997 &  0.5964 &0.9351 &0.9148   \\ \hline
		Recall Weighted     &  0.5544 &0.5522& 0.9324 & 0.9129   \\ \hline
		Top 5 Accuracy      &0.8194 & 0.8185 &0.9961 &  0.9878   \\ \hline
	\end{tabular}
	\caption{Injected Symptoms. Metric Summary}
	\label{table:tab_7}
\end{table}
\end{frame}

\begin{frame}
\frametitle{Results Discussion }
\begin{itemize}
	\item For the most of the differently generated data sets, results stay the same
	\item (An important necessary test would be to see how each model performs on data generated differently than it's test data)
	\item However, increasing the minimum number of symptoms that must be presented has a huge effect on the results
	\item This increase in the symptom count per condition translates to having more \textit{relevant} features for the models to use in decision making
	\item It also highlights the fact that with only one symptom there simply is not information to make a reasonable prediction.
	\item Will there be the same performance measure if the same minimum number of symptoms are enforced but instead randomly selecting instead of a probabilistic selection?
\end{itemize}
\end{frame}

\section{Medvice Mia Sync-Up}
\frametitle{Background }
\begin{frame}
\begin{itemize}
	\item The use of synthetic data was motivated by a lack of access to real data
	\item All the analysis done so far has been aimed at obtaining models that might be able to perform reasonably well on real data
	\item Medvice's MIA platform provides an option for access to actual patient data
	\item So it is important to be able to translate Mia output to a Synthea-generated format
\end{itemize}
\end{frame}

\frametitle{Mia Data Format }
\begin{frame}
\begin{itemize}
	\item Mia is organized around \textit{"initial complaints"}
	\item These is the symptoms for which the patient most likely is worried about.
	\item Following a logical process - very much like how a doctor might act - follow up questions are asked based on the initial symptom
	\item These questions either aim to discover more about the nature, location, duration of the initial complaint
	\item Or they aim to identify other presented symptoms or details about patient history which might help the diagnosis
	\item Based on the collected questions, a doctor might be able to make a confident first guess on the possible diagnosis
\end{itemize}
\end{frame}

\frametitle{Translating Mia to Synthea }
\begin{frame}
\begin{itemize}
	\item In creating a map from Mia to Synthea, the following simplifying assumptions have been made:
	\begin{itemize}
		\item The exported questions/complaints from Mia captures the space of all possible Mia output.
		\item While the above assumption is true now, it would change with the addition of new questions/complaints - a feature which Mia has
		\item As earlier stated, the Symcat data captures all the possible conditions/symptoms
	\end{itemize}
	\item An ideal solution might involve some use of Natural Language processing to deduce the topic matter/context of Mia questions and match them to Synthea symptoms
	\item In the meantime though, a less intelligent mapping between the complaints in Mia and Synthea symptoms was created
\end{itemize}
\end{frame}

\frametitle{Translating Mia to Synthea }
\begin{frame}
\begin{itemize}
	\item As an example, consider the question: \textit{"Wat voor soort hoest heeft u?"}
	\item Possible Mia answers are:
	\begin{itemize}
		\item drooge hoest, kriebelhoest
		\item blaf-achtige hoest, hoest met slijm
	\end{itemize}
	\item These all indicate different types of cough. In Synthea however, there are only two types of cough:
	\begin{itemize}
		\item Cough and Cough with Sputum
	\end{itemize}
	\item In the Mia-Synthea map, \textit{hoest met slijm} is mapped to \textit{cough with sputum} while all the others get mapped to cough
\end{itemize}
\end{frame}

\frametitle{Translating Mia to Synthea }
\begin{frame}
\begin{itemize}
	\item Mia is also able to capture information about the nature of a symptom
	\item Take this question as an example: \textit{Hoe voelt de buikpijn precies voor u?}
	\item Possible Mia answers are:
	\begin{itemize}
		\item stekend of scherp, zeurend en sluimerend
		\item krampend en weeïg, 
	\end{itemize}
	\item The Mia options can map to \textit{abdominal-pain}, \textit{sharp-abdominal-pain} \textit{burning-abdominal-pain} in Symcat
\end{itemize}
\end{frame}

\frametitle{Translating Mia to Synthea }
\begin{frame}
\begin{itemize}
	\item The mapping also captures information from Mia about the duration of the complaint, or actual measurements (e.g temperature in the case of Fever)
	\item Information about the duration of a complaint, or temperature can then be used together with a threshold to make further classification
	\item e.g a temperature of 39 deg. can be classified as \textit{high-fever} as opposed to just \textit{fever}
	\item Similar analysis can be made for the duration of a complaint.
\end{itemize}
\end{frame}

\frametitle{Drawbacks to this Approach}
\begin{frame}
\begin{itemize}
	\item The synthea generated data currently does not capture information about patient previous history or current medication
	\item Examples: questions about which pain killers a patient is on or if they had an operation previously on their back (in the case of back pain)
	\item Answers to these questions would be extremely helpful in a diagnosis but since not modeled by Synthea they cannot be used
\end{itemize}
\end{frame}

\frametitle{Drawbacks to this Approach}
\begin{frame}
\begin{itemize}
	\item In some cases, a \textit{nearest-neighbor} mapping is used from Mia to Synthea
	\item As an example, a question in Mia asks if patient stool is watery. The closes Synthea alternative is \textit{change in stool appearance}
	\item There is obviously some loss in information when making the mapping.
\end{itemize}
\end{frame}


%------------------------------------------------

\begin{frame}
\Huge{\centerline{The End}}
\end{frame}

%----------------------------------------------------------------------------------------

\end{document} 
%%%%%%%%%%%%%%%%%%%%%%%%%%%%%%%%%%%%%%%%%
% Beamer Presentation
% LaTeX Template
% Version 1.0 (10/11/12)
%
% This template has been downloaded from:
% http://www.LaTeXTemplates.com
%
% License:
% CC BY-NC-SA 3.0 (http://creativecommons.org/licenses/by-nc-sa/3.0/)
%
%%%%%%%%%%%%%%%%%%%%%%%%%%%%%%%%%%%%%%%%%

%----------------------------------------------------------------------------------------
%	PACKAGES AND THEMES
%----------------------------------------------------------------------------------------

\documentclass{beamer}

\mode<presentation> {

% The Beamer class comes with a number of default slide themes
% which change the colors and layouts of slides. Below this is a list
% of all the themes, uncomment each in turn to see what they look like.

%\usetheme{default}
%\usetheme{AnnArbor}
%\usetheme{Antibes}
%\usetheme{Bergen}
%\usetheme{Berkeley}
%\usetheme{Berlin}
%\usetheme{Boadilla}
%\usetheme{CambridgeUS}
%\usetheme{Copenhagen}
%\usetheme{Darmstadt}
%\usetheme{Dresden}
%\usetheme{Frankfurt}
%\usetheme{Goettingen}
%\usetheme{Hannover}
%\usetheme{Ilmenau}
%\usetheme{JuanLesPins}
%\usetheme{Luebeck}
\usetheme{Madrid}
%\usetheme{Malmoe}
%\usetheme{Marburg}
%\usetheme{Montpellier}
%\usetheme{PaloAlto}
%\usetheme{Pittsburgh}
%\usetheme{Rochester}
%\usetheme{Singapore}
%\usetheme{Szeged}
%\usetheme{Warsaw}

% As well as themes, the Beamer class has a number of color themes
% for any slide theme. Uncomment each of these in turn to see how it
% changes the colors of your current slide theme.

%\usecolortheme{albatross}
%\usecolortheme{beaver}
%\usecolortheme{beetle}
%\usecolortheme{crane}
%\usecolortheme{dolphin}
%\usecolortheme{dove}
%\usecolortheme{fly}
%\usecolortheme{lily}
%\usecolortheme{orchid}
%\usecolortheme{rose}
%\usecolortheme{seagull}
%\usecolortheme{seahorse}
%\usecolortheme{whale}
%\usecolortheme{wolverine}

%\setbeamertemplate{footline} % To remove the footer line in all slides uncomment this line
%\setbeamertemplate{footline}[page number] % To replace the footer line in all slides with a simple slide count uncomment this line

%\setbeamertemplate{navigation symbols}{} % To remove the navigation symbols from the bottom of all slides uncomment this line
}

\usepackage{graphicx} % Allows including images
\usepackage{booktabs} % Allows the use of \toprule, \midrule and \bottomrule in tables

%----------------------------------------------------------------------------------------
%	TITLE PAGE
%----------------------------------------------------------------------------------------

\title[An Update]{Differential Diagnosis: Baseline Analysis and Simulating Real Data} % The short title appears at the bottom of every slide, the full title is only on the title page

\author{Obinna Stanley Agba} % Your name
\institute[TU Delft] % Your institution as it will appear on the bottom of every slide, may be shorthand to save space
{
Delft University of Technology \\ % Your institution for the title page
\medskip
\textit{o.s.agba@student.tudelft.nl} % Your email address
}
\date{\today} % Date, can be changed to a custom date

\begin{document}

\begin{frame}
\titlepage % Print the title page as the first slide
\end{frame}

\begin{frame}
\frametitle{Overview} % Table of contents slide, comment this block out to remove it
\tableofcontents % Throughout your presentation, if you choose to use \section{} and \subsection{} commands, these will automatically be printed on this slide as an overview of your presentation
\end{frame}

%----------------------------------------------------------------------------------------
%	PRESENTATION SLIDES
%----------------------------------------------------------------------------------------

%------------------------------------------------
\section{Baseline Analysis} % Sections can be created in order to organize your presentation into discrete blocks, all sections and subsections are automatically printed in the table of contents as an overview of the talk
%------------------------------------------------
\subsection{Baseline Data Generation}
\begin{frame}
\frametitle{Sample Space}
\begin{itemize}
	\item 801 Conditions i.e diseases
	\item 376 Symptoms
	\item Assumption: these conditions and symptoms capture the "illness" space
	\item Age, Gender and Race also captured
\end{itemize}
\end{frame}

\begin{frame}
\frametitle{Baseline Data Generation}
\begin{itemize}
	\item Generate Synthea compatible modules from Symcat data
	\item Use generated modules with Synthea Generator
	\item Use Symcat data as is i.e no modifications, plug and play
	\item Generated 5 Million conditions for the baseline
\end{itemize}
\end{frame}

\subsection{Naive Bayes on Baseline} % A subsection can be created just before a set of slides with a common theme to further break down your presentation into chunks
\begin{frame}
\frametitle{Naive Bayes on Baseline}
\begin{itemize}
	\item Naive Bayes (N.B.) assumes conditional independence among features
	\item Features in this case include gender, age, race and all symptoms
	\item No hyper-parameters to be optimized
\end{itemize}
\end{frame}

\begin{frame}
\frametitle{Learning Curve (accuracy): Naive Bayes on Baseline}
\begin{figure}
	\includegraphics[width=0.7\linewidth]{figs/nb_lc_baseline.pdf}
	\caption{Learning Curve (Accuracy) for Naive Bayes}
\end{figure}
\end{frame}

\begin{frame}
\frametitle{Learning Curve (precision): Naive Bayes on Baseline}
\begin{figure}
	\includegraphics[width=0.7\linewidth]{figs/nb_lc_prec_baseline.pdf}
	\caption{Learning Curve (Precision) for Naive Bayes}
\end{figure}
\end{frame}

\begin{frame}
\frametitle{Naive Bayes on Baseline: Summary }
\begin{table}[]
	\begin{tabular}{|l|l|l|}
		\hline
		Metric               & Train  & Validation \\ \hline
		Accuracy             & 0.5897 & 0.5893     \\ \hline
		Precision Weighted   & 0.6743 & 0.6773     \\ \hline
		Recall Weighted      & 0.5897 & 0.5893     \\ \hline
		Top 5 Accuracy       & 0.8416 & 0.8411     \\ \hline
	\end{tabular}
	\caption{Baseline Naive Bayes Metric Summary}
	\label{table:tab_1}
\end{table}
\end{frame}

\subsection{Random Forest on Baseline} % A subsection can be created just before a set of slides with a common theme to further break down your presentation into chunks
\begin{frame}
\frametitle{Random Forest on Baseline}
\begin{itemize}
	\item Random Forest (RF) uses an ensemble of trees to learn different "\textit{selections}" of the data
	\item Trees vote on the most probable class
	\item Using sci-kit learn implementation - quite a number of hyper parameters to be optimized
\end{itemize}
\end{frame}

\begin{frame}
\frametitle{Random Forest on Baseline: Hyper-parameter Optimization}
\begin{itemize}
	\item Run a Grid Search on Hyper Parameter Space
	\item Aim is to maximize validation score with "\textit{reasonable}" model size (in MB) and train time (seconds)
	\item Targets for consideration must be within 1\% of best validation score
	\item Select optimal parameter combination using a weighted metric and maximize:
	$$
	metric = 2 * test\_score - 0.1*train\_time - 0.1 * model\_size 
	$$
	\item All parameters have been normalized
\end{itemize}
\end{frame}

\begin{frame}
\frametitle{Random Forest on Baseline: Hyper-parameter Optimization}
\begin{figure}
	\includegraphics[width=0.6\linewidth]{figs/rf_hyper_pareto_val_size.pdf}
	\caption{Random Forest Hyper-Parameter Optimization: Validation Score vs Model Size}
\end{figure}
\end{frame}

\begin{frame}
\frametitle{Random Forest on Baseline: Hyper-parameter Optimization}
\begin{figure}
	\includegraphics[width=0.6\linewidth]{figs/rf_hyper_pareto_val_time.pdf}
	\caption{Random Forest Hyper-Parameter Optimization: Validation Score vs Time}
\end{figure}
\end{frame}

\begin{frame}
\frametitle{Random Forest on Baseline: Hyper-parameter Optimization}
\begin{itemize}
	\item Previous slides show that larger trees which in turn take longer to train do not produce best results
	\item Also after a point, improvement in accuracy is no longer proportional to increased model size/ train time
	\item Using weighted metric, hyper-parameter set selected was within 0.5\% of best validation score but with a 32\% reduction in model size and 59\% reduction in train time.
\end{itemize}
\end{frame}

\begin{frame}
\frametitle{Learning Curve (accuracy): Random Forest on Baseline}
\begin{figure}
	\includegraphics[width=0.7\linewidth]{figs/rf_lc_baseline.pdf}
	\caption{Learning Curve (Accuracy) for Random Forest}
\end{figure}
\end{frame}

\begin{frame}
\frametitle{Learning Curve (precision): Random Forest on Baseline}
\begin{figure}
\includegraphics[width=0.7\linewidth]{figs/rf_lc_prec_baseline.pdf}
\caption{Learning Curve (Precision) for Random Forest}
\end{figure}
\end{frame}

\begin{frame}
\frametitle{Learning Curve on Baseline: Summary }
\begin{table}[]
	\begin{tabular}{|l|l|l|}
		\hline
		Metric             & Train  & Validation \\ \hline
		Accuracy           & 0.6378 & 0.5927     \\ \hline
		Precision Weighted & 0.6841 & 0.6281     \\ \hline
		Recall Weighted    & 0.6378 & 0.5927     \\ \hline
		Top 5 Accuracy     & 0.8829 & 0.8539     \\ \hline
	\end{tabular}
	\caption{Baseline Random Forest Metric Summary}
	\label{table:tab_2}
\end{table}
\end{frame}
%------------------------------------------------

\begin{frame}
\Huge{\centerline{The End}}
\end{frame}

%----------------------------------------------------------------------------------------

\end{document} 